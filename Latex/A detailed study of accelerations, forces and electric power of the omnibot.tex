% Use only LaTeX2e, calling the article.cls class and 12-point type.

\documentclass[12pt]{article}

%Cosas copiadas del paper de Iñigo
\usepackage[table]{xcolor}
\RequirePackage{siunitx}
\RequirePackage[T1]{fontenc}                        % T1 font encoding for PDFs
\RequirePackage{lmodern}                                % extended font definition
\RequirePackage{amsmath,amssymb,amsthm} % most important math stuff
\RequirePackage{a4wide}                                 % make better use of A4 paper
\RequirePackage{fancyhdr}                               % custom headers and footers
\RequirePackage{fncychap}                               % custom chapter titles
\RequirePackage{graphicx}                               % graphics
\RequirePackage{color}                                  % color
\RequirePackage{booktabs}                               % extra tabular commands
\RequirePackage[format=plain]{caption}  % improved caption format
\RequirePackage{nomencl}                                % cool nomenclature listing
\RequirePackage{makeidx}                                % create your index
\RequirePackage[printonlyused]{acronym}
\RequirePackage{ifthen}                                 % if-then commands (used in maketitle)
\RequirePackage{eso-pic}                                % picture in back/forground (used in cover)
\RequirePackage{relsize}                                % \textlarger, \textsmaller etc
%%
%%
%%%%%%%%%%%%%%%%%%%%%%%%%%%%%%%%%%%%%%%%%%%%%%%%%%%%%%%%%%%%
%%
%%  Set up Matlab and C++ Listings
%%  REQUIRES PACKAGE listings AND colortbl
%%
%%%%%%%%%%%%%%%%%%%%%%%%%%%%%%%%%%%%%%%%%%%%%%%%%%%%%%%%%%%%
%%
%%

\RequirePackage{listings}%

\usepackage[utf8]{inputenc}

\usepackage{pgf,tikz}

\usepackage{pgfgantt}

\usepackage{mathrsfs}

\usepackage{mathtools}

\usepackage{gensymb}

\usepackage{float}

\usepackage{needspace}

\usepackage{pgfplots}

\usepackage[nodayofweek,level]{datetime}

\usepackage{todonotes}

\usepackage{bm}

\usepackage{enumitem}

\usepackage{graphicx}

\usepackage{subcaption}

\usepackage{indentfirst}

\usepackage{multirow}

\usepackage{eurosym}

\usepackage{hhline}

\usepackage{esvect}


\usepackage{upgreek}
\usepackage{arydshln}
\usepackage{algorithm}
\usepackage[noend]{algpseudocode}

\usepackage[nameinlink,capitalise]{cleveref}

\usepackage{mathtools}
%%%%%%%%%%% COPYPASTEADO DE GITHUB
\usepackage{amsmath}
\usepackage{amssymb}

\usepackage{wrapfig}

%redefine vector and Real symbols for faster typing
\renewcommand{\vec}[1]{\bm{#1}}
\newcommand{\R}{\mathbb R}
\newcommand{\Z}{\mathbb Z}
\newcommand{\foralli}[1][]{\forall i \in \{1\dots n_{#1}\}}
\newcommand{\forallj}[1][]{\forall j \in \{1\dots n_{#1}\}}
\newcommand{\forallk}[1][]{\forall k \in \{1\dots n_{#1}\}}
\newcommand{\dd}[2]{\frac{\partial #1}{\partial #2}}
\newcommand{\dt}[1]{\frac{d #1}{d t}}
\newcommand{\torque}{\tau}
\newcommand{\w}{\dot\varphi}
\newcommand{\h}{\frac{1}{2}}
\newcommand{\pare}[1]{\left(#1\right)}
\newcommand{\brac}[1]{\left\{#1\right\}}

\newcommand{\mat}[2][b]{\begin{#1matrix}#2\end{#1matrix}}

%set spacing between rows in tables
\renewcommand{\arraystretch}{1.2}

\def\F{\vec F}
\def\Torque{\vec \Gamma}
\def\R{\vec R}

\def\q{\vec q}
\def\M{\vec M}
\def\I{\vec I}
\def\C{\vec C}
\def\mults{\vec \lambda}

%   Fin de la parte copypasteada
\graphicspath{ {images/} }

% Users of the {thebibliography} environment or BibTeX should use the
% scicite.sty package, downloadable from *Science* at
% www.sciencemag.org/about/authors/prep/TeX_help/ .
% This package should properly format in-text
% reference calls and reference-list numbers.

\usepackage{scicite}

% Use times if you have the font installed; otherwise, comment out the
% following line.

\usepackage{times}

% The preamble here sets up a lot of new/revised commands and
% environments.  It's annoying, but please do *not* try to strip these
% out into a separate .sty file (which could lead to the loss of some
% information when we convert the file to other formats).  Instead, keep
% them in the preamble of your main LaTeX source file.


% The following parameters seem to provide a reasonable page setup.

\topmargin 0.0cm
\oddsidemargin 0.2cm
\textwidth 16cm 
\textheight 21cm
\footskip 1.0cm


%The next command sets up an environment for the abstract to your paper.

\newenvironment{sciabstract}{%
\begin{quote} \bf}
{\end{quote}}


% If your reference list includes text notes as well as references,
% include the following line; otherwise, comment it out.

\renewcommand\refname{References and Notes}

% The following lines set up an environment for the last note in the
% reference list, which commonly includes acknowledgments of funding,
% help, etc.  It's intended for users of BibTeX or the {thebibliography}
% environment.  Users who are hand-coding their references at the end
% using a list environment such as {enumerate} can simply add another
% item at the end, and it will be numbered automatically.

\newcounter{lastnote}
\newenvironment{scilastnote}{%
\setcounter{lastnote}{\value{enumiv}}%
\addtocounter{lastnote}{+1}%
\begin{list}%
{\arabic{lastnote}.}
{\setlength{\leftmargin}{.22in}}
{\setlength{\labelsep}{.5em}}}
{\end{list}}


% Include your paper's title here

\title{A Detailed Study of Acceleration, Forces, and Electric Power of the Omnibot} 


% Place the author information here.  Please hand-code the contact
% information and notecalls; do *not* use \footnote commands.  Let the
% author contact information appear immediately below the author names
% as shown.  We would also prefer that you don't change the type-size
% settings shown here.

\author
{Siro Moreno$^{1\ast}$ \\
\\
\normalsize{$^{1}$Institut de Robótica i Informática Industrial}\\
\normalsize{dirección del IRI, Barcelona}\\
}
% Include the date command, but leave its argument blank.

\date{}



%%%%%%%%%%%%%%%%% END OF PREAMBLE %%%%%%%%%%%%%%%%



\begin{document} 

% Double-space the manuscript.

\baselineskip24pt

% Make the title.

\maketitle 



% Place your abstract within the special {sciabstract} environment.

\begin{sciabstract}
  We will discuss some considerations and insights of using an alternative axes with the Omnibot robot, rotated 45 degrees respect to the symmetry axes.
\end{sciabstract}



% In setting up this template for *Science* papers, we've used both
% the \section command and the \paragraph* command for topical
% divisions.  Which you use will of course depend on the type of paper
% you're writing.  Review Articles tend to have displayed headings, for
% which \section is more appropriate; Research Articles, when they have
% formal topical divisions at all, tend to signal them with bold text
% that runs into the paragraph, for which \paragraph* is the right
% choice.  Either way, use the asterisk (*) modifier, as shown, to
% suppress numbering.

\section{Definition of new axes}

\begin{wrapfigure}{l}{0.3\textwidth}
	\centering
	\includegraphics[width=\linewidth]{Omnibot_45_deg.png}
	\captionof{figure}{Omnibot new axes}
	\label{fig:omnibot}
\end{wrapfigure}
Let's define a new vector base system, anchored to the center of mass of the Omnibot and rotated 45 degrees clockwise respect to the B basis.
In order to describe the position of the wheels in this new base, we can define the magnitudes:
$$ l_2 = \frac{L-l}{\sqrt{2}}\ , \ L_2 = \frac{L+l}{\sqrt{2}}$$
We can follow the same process to calculate the equations of the system, but using the base 2 instead of the base B as an intermediate base between the $q_r$ and $q_w$ coordinates. We will also define $psi$ as the angle between the global X axis and the $x_2$ axis, instead of using the $x_B$ axis.

The first difference that we can observe is that using this base, some zeroes appear on the R matrix:

$$ R = \frac{\sqrt{2}}{r}\left[\begin{matrix}1 & 0 & - L_{2}\\0 & 1 & L_{2}\\0 & 1 & - L_{2}\\1 & 0 & L_{2}\end{matrix}\right]$$

We observe that movement on $x_2$ and $y_2$ uncouples.

\subsection{Definition of $\vec{w}$ and $\vec{a}$ }

Let's define two new concepts: $\vec{w}$ and $\vec{a}$:

$$ \vec{w} \equiv \left[\begin{matrix} w_1\\w_2\\\dot{\psi}\end{matrix}\right] \equiv \R_{\psi}^T \dot{\q_r}\ ,\ \vec{a} \equiv \left[\begin{matrix} a_1\\a_2\\\ddot{\psi}\end{matrix}\right] \equiv \R_{\psi}^T \ddot{\q_r}$$

We observe that  $\vec{w}$ and $\vec{a}$ are the proyection on base 2 of the global speeds and accelerations of the robot's center of mass. We must be cautious when using them, because since $\R_{\psi}$ can vary with time, in general $\frac{d\vec{w}}{dt}\neq\vec{a}$.
$$\frac{d\vec{w}}{dt} = \frac{d(\R_{\psi}^T \dot{\q_r})}{dt} =  \frac{d(\R_{\psi}^T)}{dt}\dot{\q_r} + \R_{\psi}^T \ddot{\q_r}=  \frac{d(\R_{\psi}^T)}{dt}\dot{\q_r} + \vec{a}$$

Why are then these variables useful? The answer begins with the relationship that connects  $\vec{w}$ and $\dot{\q_w}$:
$$\dot{\q_w} = \R \R_{\psi}^T \dot{\q_r} = \R \vec{w} = \frac{\sqrt{2}}{2}\left[\begin{matrix}- L_{2} \dot{\psi} + w_1\\L_{2} \dot{\psi} + w_2\\- L_{2} \dot{\psi} + w_2\\L_{2} \dot{\psi} + w_1\end{matrix}\right]$$
We can observe from this that rotation of wheels 1 and 4 only depend on $\dot{\psi}$ and $w_1$, while rotation of wheels 2 and 3 only depend on $\dot{\psi}$ and $w_2$.

This means that, as the speed of the robot is limited by the maximum speed in any of its wheels' shafts, the envelope of achievable speeds in $x_2$ and $y_2$ will always be a square whose size depends on $\dot{\psi}$:
\begin{figure}[h]
	\centering
	\includegraphics[width=.5\linewidth]{speed_envelope}
	\captionof{figure}{Speed Envelope}
	\label{fig:speed_envelope}
\end{figure}

\section{Proyection of the equation}

As we know from the work of Iñigo, we can describe the Omnibot system dynamics as
\begin{gather}
	\vec H \ddot \q_r+ \vec K \dot \q_r=\R_{\psi}\R ^T\Torque \label{eq:solution}
\end{gather}
Where:
\begin{gather}
	\vec H = \mat{ m+\frac{4\,I_w}{r^2} & 0 & 0 \\ 0 & m+\frac{4\,I_w}{r^2} & 0 \\ 0 & 0 & I_z+\frac{4\,I_w\,{\left(L+l\right)}^2}{r^2} } \label{eq:H}
	\\
	\vec K = \mat{ 0 & \frac{4\,I_w\,\dot \psi }{r^2} & 0 \\ -\frac{4\,I_w\,\dot \psi }{r^2} & 0 & 0 \\ 0 & 0 & 0 }
\end{gather}

By definition: $ \dot{\q}_2  = \R_{\psi}^T \dot{\q_r}\ ,\ \ddot{\q}_2 = \R_{\psi}^T \ddot{\q_r}$

Which means that also $ \dot{\q}_r  = \R_{\psi} \vec{w}\ ,\ \ddot{\q}_r = \R_{\psi} \vec{a}$

So we can rewrite the system's equation as
$$	\vec H \R_{\psi} \vec{a}+ \vec K \R_{\psi} \vec{w}=\R_{\psi}\R ^T\Torque$$

Multiplying the whole equation leftside by $\R_{\psi}^T$, we get:
$$ \R_{\psi}^T	\vec H \R_{\psi} \vec{a}+ \R_{\psi}^T \vec K \R_{\psi} \vec{w}=\R ^T\Torque$$
Operating, we find that $\R_{\psi}^T	\vec H \R_{\psi} = \vec{H}$ and $\R_{\psi}^T \vec{K} \R_{\psi} = \vec{K}$, so our equation proyected on axes 2 becomes:
\begin{gather}
\vec H \ddot \q_2+ \vec K \dot \q_2=\R ^T\Torque \label{eq:axes_2_eq}
\end{gather}
We can use this equation to gain more insight on the nature of the system.

\section{Electric power}
\subsection{Constant movement without rotation}
Let's start by assuming a straight uniform movement without rotation:
$$ \vec{w} = \left[\begin{matrix}w_1\\w_2\\0\end{matrix}\right]\ ,\ \dot{\q_w} = \R \vec{w} = \frac{\sqrt{2}}{r} \left[\begin{matrix}w_1\\w_2\\w_2\\w_1\end{matrix}\right]$$

We remember that the electric motor model looks like this:
$$ V = K_m N\dot{\phi} + Ri$$
$$ \tau_m = N K_ei\mu_{trans} - \tau_r$$
$$ \tau_r = a \dot{\phi} + b\operatorname{sign}(\dot{\phi}) $$
And that in straight constant movement the torque output is zero, so that the electric power consumed in a motor is 
$$ P_m = 0.301 \operatorname{sign}^{2}\left(\dot{\phi}\right) + 0.638 \operatorname{sign}\left(\dot{\phi}\right) \dot{\phi} + 0.016 \dot{\phi}^{2}$$
We can add the power consumed by all four motors to get 
$$P_{tot} = 0.6 \operatorname{sign}^{2}\left(w_1\right) + 27.05 \operatorname{sign}\left(w_1\right) w_1 + 0.6 \operatorname{sign}^{2}\left(w_2\right) + 27.05 \operatorname{sign}\left(w_2\right) w_2 + 14.17 w_1^{2} + 14.17 w_2^{2}$$

Which we can plot over $w_1$ and $w_2$: 
\begin{figure}[h]
	\centering
	\includegraphics[width=.5\linewidth]{power_map_base_2}
	\captionof{figure}{Total electric power required to keep constant speed}
	\label{fig:power_ct_speed}
\end{figure}

We can observe that movement is most efficient in the $x_2$ and $y_2$ directions, when only two motors are running, while the least efficient directions are rotated 45 degrees respect to them: the symmetry axes of the robot.
\subsection{Constant movement with rotation}
If our robot keeps a straight movement with constant rotation rate:
$$ \vec{a} = \vec{0}\ ,\ \vec{w} = \left[\begin{matrix}w_1\\w_2\\\dot{\psi}\end{matrix}\right]\ ,\ \dot{\q_w} = \R \vec{w} = \frac{\sqrt{2}}{r} \left[\begin{matrix}- L_{2} \dot{\psi} + w_1\\L_{2} \dot{\psi} + w_2\\- L_{2} \dot{\psi} + w_2\\L_{2} \dot{\psi} + w_1\end{matrix}\right]$$

Subtituting in the General equation:

$$\vec H \ddot \q_2+ \vec K \dot \q_2=\vec K \dot \q_2=\R ^T\Torque$$

Multiplying the equation leftside by $\R^{-1T}$:
$$\Torque=\R^{-1T}\vec K \dot \q_2 = \frac{\sqrt{2}I_w}{r}\left[\begin{matrix}\dot{\psi} w_2\\- \dot{\psi} w_1\\- \dot{\psi} w_1\\\dot{\psi} w_2\end{matrix}\right]$$

Now the torque is not zero, but we can use the model to get an expression that models the electric power of a motor as a function of the torque $\tau_m$ and the angular speed $\dot{\phi}$:
$$P_m = \left(\frac{K_{m} a}{K_{e} \mu} + \frac{R_{e} a^{2}}{K_{e}^{2} \mu^{2} n^{2}}\right) \dot{\phi}^{2} + \left(\frac{K_{m} b \operatorname{sign}\left(\dot{\phi}\right)}{K_{e} \mu} + \frac{K_{m} \tau_{m}}{K_{e} \mu} + \frac{2 R_{e} a b \operatorname{sign}\left(\dot{\phi}\right)}{K_{e}^{2} \mu^{2} n^{2}} + \frac{2 R_{e} a \tau_{m}}{K_{e}^{2} \mu^{2} n^{2}}\right) \dot{\phi} +$$
$$+ \frac{R_{e} b^{2} \operatorname{sign}^{2}\left(\dot{\phi}\right)}{K_{e}^{2} \mu^{2} n^{2}} + \frac{2 R_{e} b \tau_{m} \operatorname{sign}\left(\dot{\phi}\right)}{K_{e}^{2} \mu^{2} n^{2}} + \frac{R_{e} \tau_{m}^{2}}{K_{e}^{2} \mu^{2} n^{2}}$$
Or, subtituting the constants with their values:
$$ P_m = 1.881 \tau_{m}^{2} + 1.505 \tau_{m} \operatorname{sign}\left(\dot{\phi}\right) + \left(1.594 \tau_{m} + 0.638 \operatorname{sign}\left(\dot{\phi}\right)\right) \dot{\phi} + 0.301 \operatorname{sign}^{2}\left(\dot{\phi}\right) + 0.016 \dot{\phi}^{2}$$

If we add the power spent by the four motors substituting in this expression $\tau_m$ and $\dot{\phi}$ by its values functions of $q_2$, we get a function that expresses the total electric power as a function of $w_1$, $w_2$ and $\dot{\psi}$:
\begin{multline}
P_{tot} = \displaystyle 0.3 \operatorname{sign}^{2}\left(4.44 \dot{\psi} - 14.99 w_1\right) - 0.08 \operatorname{sign}\left(4.44 \dot{\psi} - 14.99 w_1\right) \dot{\psi} w_2 + \\+4.01 \operatorname{sign}\left(4.44 \dot{\psi} - 14.99 w_1\right) \dot{\psi} - 13.52 \operatorname{sign}\left(4.44 \dot{\psi} - 14.99 w_1\right) w_1 + 0.3 \operatorname{sign}^{2}\left(4.44 \dot{\psi} + 14.99 w_1\right) +\\+ 0.08 \operatorname{sign}\left(4.44 \dot{\psi} + 14.99 w_1\right) \dot{\psi} w_2 + 4.01 \operatorname{sign}\left(4.44 \dot{\psi} + 14.99 w_1\right) \dot{\psi} + 13.52 \operatorname{sign}\left(4.44 \dot{\psi} + 14.99 w_1\right) w_1 +\\+ 0.3 \operatorname{sign}^{2}\left(4.44 \dot{\psi} - 14.99 w_2\right) + 0.08 \operatorname{sign}\left(4.44 \dot{\psi} - 14.99 w_2\right) \dot{\psi} w_1 + 4.01 \operatorname{sign}\left(4.44 \dot{\psi} - 14.99 w_2\right) \dot{\psi} -\\- 13.52 \operatorname{sign}\left(4.44 \dot{\psi} - 14.99 w_2\right) w_2 + 0.3 \operatorname{sign}^{2}\left(4.44 \dot{\psi} + 14.99 w_2\right) - 0.08 \operatorname{sign}\left(4.44 \dot{\psi} + 14.99 w_2\right) \dot{\psi} w_1 +\\+ 4.01 \operatorname{sign}\left(4.44 \dot{\psi} + 14.99 w_2\right) \dot{\psi} + 13.52 \operatorname{sign}\left(4.44 \dot{\psi} + 14.99 w_2\right) w_2 + 0.01 \dot{\psi}^{2} w_1^{2} + 0.01 \dot{\psi}^{2} w_2^{2} +\\+ 2.49 \dot{\psi}^{2} + 14.17 w_1^{2} + 14.17 w_2^{2}
\end{multline}
While certainly not very beautiful, this expression contains a lot of information about the influence of the parameters in power consumption. For example, plotting the power over $w_1$ and $w_2$ for different $\dot{\psi}$:
\begin{figure}[h]
	\centering
	\includegraphics[width=.9\linewidth]{power_map_base_2_psi_dot}
	\captionof{figure}{Total electric power required to keep constant speed}
	\label{fig:power_ct_speed_psi_dot}
\end{figure}

We can also try to understand beter this 3-D relationship by slicing through a plane where $w_2=0$ or where $w_1 = w_2$:
\begin{figure}[h]
	\centering
	\begin{minipage}{.5\textwidth}
		\centering
		\includegraphics[width=.9\linewidth]{power_map_y_0}
		\label{fig:power_y_0}
	\end{minipage}%
	\begin{minipage}{.5\textwidth}
		\centering
		\includegraphics[width=.9\linewidth]{power_map_y_x}
		\label{fig:power_y_x}
	\end{minipage}
\end{figure}

We can use a combination of both kinds of graphs to see, for example, how power compsumption slowly becomes independent of direction as $\dot{\psi}$ increases, or how the two conditions that spend maximum power are when all the four wheels turn at their maximum speed, either moving the robot in a straight line or rotating in place. We can also check that when $\dot{\psi} = 0$, for any given $w_1$, the power compsumption when $w_1 = w_2$ is exactly twice as when $w_2=0$, even though the total speed achived is only $\sqrt{2}$ times larger.


\section{Motors electric models and their meaning}

\begin{center}
	\begin{tabular}{ | m{12em} | m{10em}| } 
		\hline
		Complete model & Simplified model \\ 
		\hline
		$$ V = K_m N\dot{\phi} + Ri$$
		$$ \tau_m = N K_ei\mu_{trans} - \tau_r$$
		$$ \tau_r = a \dot{\phi} + b \operatorname{sign}(\dot{\phi}) $$ &
		$$ V = K_m N\dot{\phi} + Ri$$
		$$ \tau_m = N K_ei$$ \\ 
		\hline
	\end{tabular}
\end{center}
On the first column we have the electric motor model as we have been using it. It includes the effects of friction and transmission losses up to the wheel shaft. On the right we have a simplified model, that drops the transmission efficiency and friction, but keeps the electrical resistance. The reason to use this model is that we want to separate the dissipative effects that are due friction from those that are intrinsic to electric motors, and compare them to get a better understanding of the nature of the model.

We can easily obtain that:
$$P_{e, simp} = Vi = \frac{K_{m} \tau_{m} \dot{\phi}}{K_{e}} + \frac{R_{e} \tau_{m}^{2}}{K_{e}^{2} N^{2}}$$

We can observe that if we were to consider a perfect motor where resistance was zero and $K_e = K_m$, the electric power would collapse back to $P = Vi = \tau_{m} \dot{\phi}$, a perfect conversion of electricity into mechanic power. Since we have already studied the robot's mechanics without motors, no additional insight would be obtained. Thus, we are keeping the electrical losses. We can substitute the parameters with their values for additional clarity:

$$ P_{e, simp} = 0.677 \tau_{m}^{2} + 0.934 \tau_{m} \dot{\phi}$$

The equivalent expression with the complete model is more complex, but it is worth showing for the sake of symmetry:
$$ P_{e,comp} = 1.881 \tau_{m}^{2} + 1.505 \tau_{m} \operatorname{sign}\left(\dot{\phi}\right) + \left(1.594 \tau_{m} + 0.638 \operatorname{sign}\left(\dot{\phi}\right)\right) \dot{\phi} + 0.301 \operatorname{sign}^{2}\left(\dot{\phi}\right) + 0.016 \dot{\phi}^{2}$$

We can see that both expressions are functions only of $\tau_{m}$ and $\dot{\phi}$, so it will be interesting to plot them in a 2d graph of both magnitudes as axes. In order to plot only valid values, we need to calculate the envelope of achievable values, that is, maximum and minimum torque at a certain  $\dot{\phi}$. We do so by substituting the value of $V$ by its maximum and minimum value. We are interested in behaviour obtainable in normal conditions, so we will limit our graph in $\dot{\phi}$ only up to the maximum speed that the motor can achieve by itself. Since the expression is symmetric, we can start our graph at zero speed.
\begin{figure}[h]
	\centering
	\includegraphics[width=.9\linewidth]{motor_electric_power_w_and_wout_fricc}
	\captionof{figure}{power as function of torque and speed}
	\label{fig:power_f_tau_speed}
\end{figure}

Just at first sight we can spot some differences between the two graphs. The main difference is that without friction, the achievable torque is almost double, in both signs. The maximum speed achievable is slightly decreased, but the electric power looks scaled with the torque, so its limit values are not that different.

We can also see some very interesting features on this graph. Lets start with the lines where $P = 0$. 

The horizontal one, which overlaps with the zero torque axe in the no friction graph, represents the condition where $i = 0$. It is easy to understand that the electric power is zero, because it is equivalent of having a motor with its terminals disconnected. In the complete model graph, this line is also present but slightly under the axe, as it shows the friction acting as a small breaking torque.

We have another line of power zero, this time diagonal, and always equidistant from the maximum and minimum torque at any $\dot{\phi}$. This line corresponds to $V = 0$, and is also easy to understand that the electric power spent there is zero, as it is equivalent to have a motor with its terminals connected with each other.

Between them, we have a triangular area where the power is negative, that is, the motor is acting as generator. While it is expected in the simplified case, we have to be very cautious when using the complete expression. The most obvious problem we encounter is that the torque efficiency $\mu_{trans}$ is defined when the motor is driving the wheel but not the opposite! Without further testing, it is difficult to know if its value would flip to its inverse or even decay to infinite because of imposibility to run the transmission backwards, as in a worm gear. My suggestion in this case would be to take a value of zero when a negative power is calculated.
\begin{center}
	\begin{tabular}{ | m{14em} | } 
		\hline
		Modified model \\ 
		\hline
		$$ V = K_m N\dot{\phi} + Ri$$
		$$ \tau_m = N K_ei\mu_{trans} - \tau_r$$
		$$ \tau_r = a \dot{\phi} + b \operatorname{sign}(\dot{\phi}) $$
		$$ P_{e}=\begin{cases}
			Vi \text{  if  } Vi\geq 0\\
			0 \text{  if  } Vi < 0
		\end{cases}$$\\
		
		\hline
	\end{tabular}
\end{center}


\section{Application to the complete robot model}

We remember that the complete model projected on axes 2 is:
$$\vec{H} \vec a+\vec K\vec w=\R^T\vec\Gamma$$
We can explore how the proposed motor models interact with the robot model in different conditions.

\subsection{Envelope of achievable accelerations}
If we express acceleration as a function of $\Gamma$ and $\vec w$:
$$ \vec a = \left[\begin{matrix}a_{1}\\a_{2}\\\ddot{\psi}\end{matrix}\right] = \vec H^{-1}(\R^T \vec \Gamma - \vec K \vec w) =  \left[\begin{matrix}\frac{- 4 I_{w} w_{2} \dot{\psi} + \sqrt{2} r \left(\tau_{1} + \tau_{4}\right)}{4 I_{w} + m r^{2}}\\\frac{4 I_{w} w_{1} \dot{\psi} + \sqrt{2} r \left(\tau_{2} + \tau_{3}\right)}{4 I_{w} + m r^{2}}\\- \frac{\sqrt{2} L_{2} r \left(\tau_{1} - \tau_{2} + \tau_{3} - \tau_{4}\right)}{8 I_{w} L_{2}^{2} + I_{z} r^{2}}\end{matrix}\right]$$

We can observe some interesting details:
\begin{itemize}
	\item As we expected, acceleration in direction $a_1$ only depends on $\tau_1$ and $\tau_4$, while in direction $a_2$ only on $\tau_2$ and $\tau_3$. This means that when representing the envelope of achievable accelerations for any given condition, it will always be a rectangle.
	\item Maximum acceleration in direction $a_1$ is achieved with maximum $\tau_1$ and $\tau_4$, that is, with maximum voltage on those motors, and equivalent for minimum acceleration. The same logic applies to $a_2$. This means that we can easily calculate its extreme values by substituting maximum or minimum values of voltage in the motor models.
\end{itemize}
\subsection{No rotation condition}
When $\dot\psi = 0, \vec K =\vec 0$:
$$ \vec H\vec a = \R^T\vec \Gamma$$
$$\vec \Gamma = \R^{-1T}\vec H\vec a = \frac{\sqrt{2} r \left(\frac{4 I_{w}}{r^{2}} + m\right)}{4}\left[\begin{matrix}a_{1} \\a_{2} \\a_{2} \\a_{1} \end{matrix}\right]$$
$$\dot{\q_w} = \R \vec{w} = \frac{\sqrt{2}}{r} \left[\begin{matrix}w_1\\w_2\\w_2\\w_1\end{matrix}\right]$$

For each motor we can calculate the power as a function of $\tau$ and $\dot\phi$, and we saw how they can be expressed in terms of $\vec a$ and $\vec w$. Adding the power of the four motors, we get the total power.

Doing so, we can get readable formulas for the simplified and complete model, but not for the modified model, where we will switch to a numeric approach:
$$P_{e, simp} = 0.25 a_{1}^{2} + 16.95 a_{1} w_{1} + 0.25 a_{2}^{2} + 16.95 a_{2} w_{2}$$
$$P_{e, comp} = 0.69 a_{1}^{2} + 1.29 a_{1} \operatorname{sign}\left(w_{1}\right) + 0.69 a_{2}^{2} + 1.29 a_{2} \operatorname{sign}\left(w_{2}\right) + 14.17 w_{1}^{2} +$$
$$+ 42.41 w_{1} \left(0.68 a_{1} + 0.64 \operatorname{sign}\left(w_{1}\right)\right) + 14.17 w_{2}^{2} + 42.41 w_{2} \left(0.68 a_{2} + 0.64 \operatorname{sign}\left(w_{2}\right)\right) +$$
$$+ 0.6 \operatorname{sign}^{2}\left(w_{1}\right) + 0.6 \operatorname{sign}^{2}\left(w_{2}\right)$$
\subsubsection{Envelope of accelerations}
For both models, simple and complete, we can find an expression that give the value of torque as a function of the speed and the voltage:
$$\tau_{simp} = \displaystyle - \frac{K_{e} K_{m} N^{2} \dot{\phi}}{R_{e}} + \frac{K_{e} N V}{R_{e}} \approx 0.894 V - 1.38 \dot{\phi}$$
$$\tau_{comp} = - \frac{K_{e} K_{m} N^{2} \mu \dot{\phi}}{R_{e}} + \frac{K_{e} N V \mu}{R_{e}} - a \dot{\phi} - b \operatorname{sign}\left(\dot{\phi}\right)\approx 0.537 V - 0.4 \operatorname{sign}\left(\dot{\phi}\right) - 0.838 \dot{\phi}$$

We can substitute in these expressions the value of V with it maximum (24V), minimum (-24V) and zero values, in order to get maximum, minimum and zero-voltage torques as functions of $\dot \phi$. We can then get arrays for maximum, minimum and zero-voltage $\Gamma$ as functions of $\vec w$. Then, using
$$\vec a = \vec H^{-1} \R^{T} \vec \Gamma$$ 
We can obtain values for maximum, minimum and zero-voltage accelerations:
$$\vec a_{simp, max} \approx \left[\begin{matrix}50.178 - 68.379 w_{1}\\50.178 - 68.379 w_{2}\\0\end{matrix}\right], \  \vec a_{simp, min} \approx \left[\begin{matrix}- 68.379 w_{1} - 50.178\\- 68.379 w_{2} - 50.178\\0\end{matrix}\right], \  \vec a_{simp, 0V} \approx \left[\begin{matrix}- 68.379 w_{1}\\- 68.379 w_{2}\\0\end{matrix}\right]$$
$$\vec a_{comp, max} \approx \left[\begin{matrix}- 41.523 w_{1} - 0.935 \operatorname{sign}\left(w_{1}\right) + 30.107\\- 41.523 w_{2} - 0.935 \operatorname{sign}\left(w_{2}\right) + 30.107\\0\end{matrix}\right]$$
$$\vec a_{comp, min} \approx  \left[\begin{matrix}- 41.523 w_{1} - 0.935 \operatorname{sign}\left(w_{1}\right) - 30.107\\- 41.523 w_{2} - 0.935 \operatorname{sign}\left(w_{2}\right) - 30.107\\0\end{matrix}\right], \ \vec a_{comp, 0V} \approx   \left[\begin{matrix}- 41.523 w_{1} - 0.935 \operatorname{sign}\left(w_{1}\right)\\- 41.523 w_{2} - 0.935 \operatorname{sign}\left(w_{2}\right)\\0\end{matrix}\right]$$

We can observe than in both models, $a_{1,max} - a_{1,min} = a_{2,max} - a_{2,min} = C_{model}$, being $C_{model}$ a constant value different in each model. We can also see that the zero-voltage value is always the mean value between the maximum and the minimum. This means, as we will see in the graphs, that the envelope of achievable accelerations will always be a square of side $C_{model}$, centered on $\vec a_{0V}$.
\subsubsection{Movement in one axis}
Let's start with a simple case where $w_2$ and $a_2$ are zero. In this case, only motors 1 and 4 work, and both run at the same speed and equal force. We can plot the electric power needed to achieve a certain acceleration while moving at a given speed, and we will see that it is the same graph as the motor alone, but with the power doubled since we now have two motors running.

\begin{figure}[h]
	\centering
	\includegraphics[width=1\linewidth]{total_electric_power_w_and_wout_fricc}
	\captionof{figure}{power as function of acceleration and speed}
	\label{fig:power_f_a1_w1}
\end{figure}

\subsubsection{Acceleration from static}
Let's suppose that we are static $\vec{w} = \vec{0}$ and want to get an acceleration. Substituting in the general no rotation expressions, we get:
$$P_{e, simp} = 0.25 a_{1}^{2} + 0.25 a_{2}^{2}$$
$$P_{e, comp} = 0.69 a_{1}^{2} + 0.69 a_{2}^{2}$$

We can see that in both cases, power required to accelerate scales with the acceleration module squared, independent from the direction. The modified model is equal to the complete model because there is no condition with negative power in the envelope of feasible accelerations.
\begin{figure}[h]
	\centering
	\includegraphics[width=1\linewidth]{power_from_static}
	\captionof{figure}{power as function of acceleration, from static}
	\label{fig:power_from_static}
\end{figure}

\subsubsection{Moving at half speed in $w_1$}
As a second example, let's repeat the analysis but supposing the robot is moving at half its maximum speed in the direction of $w_1$.
\begin{figure}[h]
	\centering
	\includegraphics[width=1\linewidth]{power_from_w1}
	\captionof{figure}{power as function of acceleration, moving in $w_1$}
	\label{fig:power_from_w1}
\end{figure}
We can see that in the three graphs, the size of the square envelope of achievable accelerations remains unchanged, but its center is displaced to the point that corresponds to switching all voltages to zero.

Since we are moving with pure $w_1$ speed, the motors 1 and 4 have a range of torques where they could generate power (in the models that allow it). In the simple and complete models, we can observe it as a circular region that appears where power is negative, that is, where more power is generated in a pair of motors than spent in the other two.

In the modified model, as expected, that region collapses in a line. We can observe a rectangular area is created between the two $a_1$ values that limit where the power of motors 1 and 4 are zero. Since in that region of the map the power only depends on $a_2$, the lines become horizontal.

\subsubsection{Moving at half speed in both $w_1$ and $w_2$}
As a third example, let's repeat the analysis but supposing the robot is moving at half its maximum speed in both directions.

\begin{figure}[h]
	\centering
	\includegraphics[width=1\linewidth]{power_from_w1_w2}
	\captionof{figure}{power as function of acceleration, moving in $w_1$ and $w_2$}
	\label{fig:power_from_w1_w2}
\end{figure}

Again, we can observe that the envelope remains unchanged in size and centered in the zero voltage point for the three graphs.

In the modified model, we can observe that the effect of degeneration into straight lines now applies for both axes: in the range of $a_1$ where power of motors 1 and 4 is zero, lines are horizontal, while in the range of $a_2$ where power of motors 2 and 3 is zero, lines are vertical. In the area in which both happen at the same time, we have a rectangle where power spent is zero.

\subsection{Considering rotation}
Let's consider a general situation in which apart from its speed in directions $w_1$ and $w_2$, the robot is rotating at a rate $\dot(psi)$. We have the three model expressions that link the electric power as function of $\vec{\Gamma}$ and $\dot\q_w$, but in order to get it as a function of $\dot{\q_r}$ and $\ddot{\q_r}$ we have to take on account additional terms:
$$ \vec H\vec a+\vec K\vec w=\R^T\Gamma$$
$$ \vec\Gamma = \R^{-T}(\vec H\vec a+\vec K\vec w)$$
$$ \vec\Gamma = \frac{\sqrt{2}}{8 L_{2} r} \left[\begin{matrix}2 L_{2} \left(4 I_{w} w_{2} \dot{\psi} + a_{1} \left(4 I_{w} + m r^{2}\right)\right) - \left(8 I_{w} L_{2}^{2} + I_{z} r^{2}\right) \ddot{\psi}\\- 2 L_{2} \left(4 I_{w} w_{1} \dot{\psi} - a_{2} \left(4 I_{w} + m r^{2}\right)\right) + \left(8 I_{w} L_{2}^{2} + I_{z} r^{2}\right) \ddot{\psi}\\- 2 L_{2} \left(4 I_{w} w_{1} \dot{\psi} - a_{2} \left(4 I_{w} + m r^{2}\right)\right) - \left(8 I_{w} L_{2}^{2} + I_{z} r^{2}\right) \ddot{\psi}\\2 L_{2} \left(4 I_{w} w_{2} \dot{\psi} + a_{1} \left(4 I_{w} + m r^{2}\right)\right) + \left(8 I_{w} L_{2}^{2} + I_{z} r^{2}\right) \ddot{\psi}\end{matrix}\right]$$
$$\vec\Gamma \approx \left[\begin{matrix}0.428 a_{1} + 0.056 w_{2} \dot{\psi} - 0.035 \ddot{\psi}\\0.428 a_{2} - 0.056 w_{1} \dot{\psi} + 0.035 \ddot{\psi}\\0.428 a_{2} - 0.056 w_{1} \dot{\psi} - 0.035 \ddot{\psi}\\0.428 a_{1} + 0.056 w_{2} \dot{\psi} + 0.035 \ddot{\psi}\end{matrix}\right] $$
$$\dot{\q_w} = \frac{\sqrt{2}}{2}\left[\begin{matrix}- L_{2} \dot{\psi} + w_1\\L_{2} \dot{\psi} + w_2\\- L_{2} \dot{\psi} + w_2\\L_{2} \dot{\psi} + w_1\end{matrix}\right] \approx \left[\begin{matrix}21.2 w_{1} - 6.28 \dot{\psi}\\21.2 w_{2} + 6.28 \dot{\psi}\\21.2 w_{2} - 6.28 \dot{\psi}\\21.2 w_{1} + 6.28 \dot{\psi}\end{matrix}\right]$$

Connecting these expressions through the three different motor models allows us to express the total electric power as a function of the robot speeds and accelerations. Only in the case of the simple model is the resulting expression clear enough to be useful: 
$$ P_{e, simp} \approx 0.25 a_{1}^{2} + 16.95 a_{1} w_{1} + 0.07 a_{1} w_{2} \dot{\psi} + 0.25 a_{2}^{2} - 0.07 a_{2} w_{1} \dot{\psi} + 16.95 a_{2} w_{2} + 0.82 \dot{\psi} \ddot{\psi} $$

\subsubsection{Acceleration enelope}
Let's consiger again the general equation and express acceleration as a function of torques:
$$ \vec H\vec a+\vec K\vec w=\R^T\Gamma$$
$$ \vec a = \vec H^{-1}(\R^T\vec\Gamma - \vec K \vec w) \approx \left[\begin{matrix}\frac{- 4 I_{w} w_{2} \dot{\psi} + \sqrt{2} r \left(\tau_{1} + \tau_{4}\right)}{4 I_{w} + m r^{2}}\\\frac{4 I_{w} w_{1} \dot{\psi} + \sqrt{2} r \left(\tau_{2} + \tau_{3}\right)}{4 I_{w} + m r^{2}}\\- \frac{\sqrt{2} L_{2} r \left(\tau_{1} - \tau_{2} + \tau_{3} - \tau_{4}\right)}{8 I_{w} L_{2}^{2} + I_{z} r^{2}}\end{matrix}\right]$$
This expression gives us a very important clue: Maximum and minimum acceleration in direction $w_1$ will always correspond to maximum and minimum torque in motors 1 and 4, and the same can be said about direction $w_2$ and motors 2 and 3. That means that in order to find the acceleration envelope, we can safely substitute in this expression with the values of maximum and minimum voltages.

For the simple model, we get:
$$\vec a_{max, simp} \approx \left[\begin{matrix}- 68.379 w_{1} - 0.132 w_{2} \dot{\psi} + 50.178\\0.132 w_{1} \dot{\psi} - 68.379 w_{2} + 50.178\\- 247.361 \dot{\psi}\end{matrix}\right]$$
$$\vec a_{min, simp} \approx \left[\begin{matrix}- 68.379 w_{1} - 0.132 w_{2} \dot{\psi} - 50.178\\0.132 w_{1} \dot{\psi} - 68.379 w_{2} - 50.178\\- 247.361 \dot{\psi}\end{matrix}\right]$$
$$\vec a_{0V, simp} \approx \left[\begin{matrix}- 68.379 w_{1} - 0.132 w_{2} \dot{\psi}\\0.132 w_{1} \dot{\psi} - 68.379 w_{2}\\- 247.361 \dot{\psi}\end{matrix}\right]$$

We can observe that the difference between the maximum and minimum values remains a constant for any condition, and again te zero-voltage condition is the mean between the two. This means that the acceleration envelope will always be a square centered on the zero-voltage value.

We observe also that the value of $\ddot \psi$ is the same for the three conditions, a negative constant times its current rotational speed. This doesn't mean that this is the only value it can have, but rather that focusing all the power in accelerating in the plane has the same braking effect on the spinning as putting the four motors at zero voltage. We will use this in our graphs, because it represents a constant value of $\ddot \psi$ that we know that we can achieve over the whole area of the envelope. We will consider the relationship between the envelope of accelerations and  $\ddot \psi$ in a later point.

For the complete model, the same conclusions can be achieved, even though  the expressions are quite complicated. Interested readers can find them in Annex 1 
\subsubsection{Influence of $\dot \psi$ on power}
We can now plot different combinations of $\dot \psi$, $w_1$ and $w_2$ in order to observe the influence that spinning has on power, Both in the cas of moving at half speed in the direction of $w_1$ in figure \ref{fig:psidot_comp_w1}, and moving at half speed in both directions of $w_1$ and $w_2$ in figure \ref{fig:psidot_comp_w1_w2} : 

In order to be sure that the conditions that we plot are feasible for the robot by its own means, we have to make sure that:
$$|\frac{\dot \psi}{\dot \psi_{max}}| + |\frac{w_1}{w_{1,max}}| \leq 1 $$
$$|\frac{\dot \psi}{\dot \psi_{max}}| + |\frac{w_2}{w_{2,max}}| \leq 1 $$

\begin{figure}[h]
	\centering
	\includegraphics[width=1\linewidth]{psidot_comp_w1}
	\captionof{figure}{power as function of acceleration, moving in $w_1$}
	\label{fig:psidot_comp_w1}
\end{figure}
\begin{figure}[h]
	\centering
	\includegraphics[width=1\linewidth]{psidot_comp_w1_w2}
	\captionof{figure}{power as function of acceleration, moving in $w_1$ and $w_2$}
	\label{fig:psidot_comp_w1_w2}
\end{figure}

In both conditions, we can see that the power requisites are not very different for the simple and complete motor models. If we plot the difference in power between the rotation and non-rotation conditions, we can observe that the difference in power is usually between one and three orders of magnitude smaller than the total power. This is shown for movement in direction of $w_1$ in figure \ref{fig:psidot_comp_w1_diff} and for movement in both directions in figure \ref{fig:psidot_comp_w1_w2_diff}.
\begin{figure}[h]
	\centering
	\includegraphics[width=1\linewidth]{psidot_comp_w1_diff}
	\captionof{figure}{Difference in power as function of acceleration, moving in $w_1$}
	\label{fig:psidot_comp_w1_diff}
\end{figure}
\begin{figure}[h]
	\centering
	\includegraphics[width=1\linewidth]{psidot_comp_w1_w2_diff}
	\captionof{figure}{Difference in power as function of acceleration, moving in $w_1$ and $w_2$}
	\label{fig:psidot_comp_w1_w2_diff}
\end{figure}

We can also observe that the modified model gives us shapes that at first glance seem irregular and strange. However, when we plot it in detail, and compare it with the values of the acceleration at which different motors enter the zero-power area, the shape becomes understandable. In figure \ref{fig:comparacion_modelos}, we use a condition where $w_1$ is half the maximum speed and $w_2$, a third. The horizontal lines are the values of $a_2$ at which motors 2 and 3 enter their zero-power area, and vertical lines are the same for motors 1 and 4. Looking closely, we can see that the estrange shapes are a combination of circles, were all motors work normally, ellipses, where only one motor is in the zero-power area, and straight lines, where two motors are there. The outlines of the equivalent conditions for the complete model are shown also as dotted lines in order to help visualization of their differences.

\begin{figure}[h]
	\centering
	\includegraphics[width=1\linewidth]{comparacion_modelos}
	\captionof{figure}{Power as function of acceleration, showing the difference between complete and modified models, moving in $w_1$ and $w_2$}
	\label{fig:comparacion_modelos}
\end{figure}

\section{Conclusions}

We have used the new axes base 2 to gain meaningful insights and understandings of how the Omnibot works and moves.
\section{Annex 1: complicated expressions}

Maximum, minimum and zero-voltage accelerations for complete model:
\tiny
$$\left[\begin{matrix}- 41.523 w_{1} - 0.132 w_{2} \dot{\psi} - 0.467 \operatorname{sign}\left(21.203 w_{1} - 6.285 \dot{\psi}\right) - 0.467 \operatorname{sign}\left(21.203 w_{1} + 6.285 \dot{\psi}\right) + 30.107\\0.132 w_{1} \dot{\psi} - 41.523 w_{2} - 0.467 \operatorname{sign}\left(21.203 w_{2} - 6.285 \dot{\psi}\right) - 0.467 \operatorname{sign}\left(21.203 w_{2} + 6.285 \dot{\psi}\right) + 30.107\\2.853 \operatorname{sign}\left(21.203 w_{1} - 6.285 \dot{\psi}\right) - 2.853 \operatorname{sign}\left(21.203 w_{1} + 6.285 \dot{\psi}\right) + 2.853 \operatorname{sign}\left(21.203 w_{2} - 6.285 \dot{\psi}\right) - 2.853 \operatorname{sign}\left(21.203 w_{2} + 6.285 \dot{\psi}\right) - 150.209 \dot{\psi}\end{matrix}\right]$$
$$ \left[\begin{matrix}- 41.523 w_{1} - 0.132 w_{2} \dot{\psi} - 0.467 \operatorname{sign}\left(21.203 w_{1} - 6.285 \dot{\psi}\right) - 0.467 \operatorname{sign}\left(21.203 w_{1} + 6.285 \dot{\psi}\right) - 30.107\\0.132 w_{1} \dot{\psi} - 41.523 w_{2} - 0.467 \operatorname{sign}\left(21.203 w_{2} - 6.285 \dot{\psi}\right) - 0.467 \operatorname{sign}\left(21.203 w_{2} + 6.285 \dot{\psi}\right) - 30.107\\2.853 \operatorname{sign}\left(21.203 w_{1} - 6.285 \dot{\psi}\right) - 2.853 \operatorname{sign}\left(21.203 w_{1} + 6.285 \dot{\psi}\right) + 2.853 \operatorname{sign}\left(21.203 w_{2} - 6.285 \dot{\psi}\right) - 2.853 \operatorname{sign}\left(21.203 w_{2} + 6.285 \dot{\psi}\right) - 150.209 \dot{\psi}\end{matrix}\right]$$
$$\left[\begin{matrix}- 41.523 w_{1} - 0.132 w_{2} \dot{\psi} - 0.467 \operatorname{sign}\left(21.203 w_{1} - 6.285 \dot{\psi}\right) - 0.467 \operatorname{sign}\left(21.203 w_{1} + 6.285 \dot{\psi}\right)\\0.132 w_{1} \dot{\psi} - 41.523 w_{2} - 0.467 \operatorname{sign}\left(21.203 w_{2} - 6.285 \dot{\psi}\right) - 0.467 \operatorname{sign}\left(21.203 w_{2} + 6.285 \dot{\psi}\right)\\2.853 \operatorname{sign}\left(21.203 w_{1} - 6.285 \dot{\psi}\right) - 2.853 \operatorname{sign}\left(21.203 w_{1} + 6.285 \dot{\psi}\right) + 2.853 \operatorname{sign}\left(21.203 w_{2} - 6.285 \dot{\psi}\right) - 2.853 \operatorname{sign}\left(21.203 w_{2} + 6.285 \dot{\psi}\right) - 150.209 \dot{\psi}\end{matrix}\right]$$

% Your references go at the end of the main text, and before the
% figures.  For this document we've used BibTeX, the .bib file
% scibib.bib, and the .bst file Science.bst.  The package scicite.sty
% was included to format the reference numbers according to *Science*
% style.


%\bibliography{scibib}

%\bibliographystyle{Science}



% Following is a new environment, {scilastnote}, that's defined in the
% preamble and that allows authors to add a reference at the end of the
% list that's not signaled in the text; such references are used in
% *Science* for acknowledgments of funding, help, etc.

%\begin{scilastnote}
%\item We've included in the template file \texttt{scifile.tex} a new
%environment, \texttt{\{scilastnote\}}, that generates a numbered final
%citation without a corresponding signal in the text.  This environment
%can be used to generate a final numbered reference containing
%acknowledgments, sources of funding, and the like, per {\it Science\/}
%style.
%\end{scilastnote}




% For your review copy (i.e., the file you initially send in for
% evaluation), you can use the {figure} environment and the
% \includegraphics command to stream your figures into the text, placing
% all figures at the end.  For the final, revised manuscript for
% acceptance and production, however, PostScript or other graphics
% should not be streamed into your compliled file.  Instead, set
% captions as simple paragraphs (with a \noindent tag), setting them
% off from the rest of the text with a \clearpage as shown  below, and
% submit figures as separate files according to the Art Department's
% instructions.


%\clearpage

%\noindent {\bf Fig. 1.} Please do not use figure environments to set
%up your figures in the final (post-peer-review) draft, do not include graphics in your
%source code, and do not cite figures in the text using \LaTeX\
%\verb+\ref+ commands.  Instead, simply refer to the figure numbers in
%the text per {\it Science\/} style, and include the list of captions at
%the end of the document, coded as ordinary paragraphs as shown in the
%\texttt{scifile.tex} template file.  Your actual figure files should
%be submitted separately.



\end{document}




















